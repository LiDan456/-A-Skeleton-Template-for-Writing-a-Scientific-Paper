\begin{document}

\title{A Skeleton Template for Writing a Scientific Paper with Murmuring}
% First edition by Dan Li 2023.06.13

% Murmuring Points
% 0. Name the main tex file as "main_xxx.tex".
% 1. Do not use 1,2,3 to name the labels of your references/figures/tables/equatuions. Label names should be a shot hint about the content, so that you can recall it whereever you want to refer it.
% 2. Never use any concepts/nouns that you are not sure to describe your work. It is better to use plain but correct vocabulary than use trendy/advanced but inaccurate vocabulary. 
% 3. Carefully choose the adjectives you used to describe your contributions and proposed method. These decriptions should be proven with numerical results.
% 4. A good research story is ussually like: in a specific field, researchers are doing what kind of research/applicatuions-->you find an unsolved problem (either theoretical or practical)-->there are several dlearly described challenges to solve this problem-->describe how to address these challenges and solve the problem.
% 5. Carefully motivate the work with insights into the details about algorithms/models/systems.
% 6. Use research questions as clus to link your writings about challenges, contributions experiment set-up and results.
% 7. Never use any Translation tool (Google/Baidu/Youdao...) to translate a whole paragraph. Just write with simple but logical English. If you are really headic about English writing, pass me a report in Chinese (but this is not recommended...).


\begin{Abstract}
    % 1-2 sentences for background introduction of the chosen topic.
    % 1-2 sentences for the problem not solved by existing works. Motivate your work based on the shortcomings of existing algorithms/models or a new angle to consider the problem you want to solve.
    % 2-3 sentences to describe your method.
    % 1-2 sentences to summarize your experiment and highlight your results and findings.
\end{abstract}

\begin{Key Words}
% key words should be highly related to the topic and the proposed method. They could be phrases describing the topic, and key techniques adoptted in this work.

\end{Key Words}

\section{Introduction}
% 1 paragraph for background on the meanings of your chosen topic/problem.

% 1 paragraph to describe the problem you want to solve based on literature or practical experiences (better to combine both). This is your motivation for writing this paper.

% 1-3 paragraphs to describe challenges. Based on existing works, talk about what has not been solved/considered but should be solved/considered for better performance/field improvement/application benefit. Here, you can briefly propose some research questions based on the challenges.

% 1 paragraph to describe your proposed method (need to correspond to the challenges mentioned previously) and contributions. Describe the method by stating how you solve the challenges. One contribution connects one challenge you solved. The last contribution could be your experimental results and findings.

% 1 paragraph to summarize the remaining organization of this paper.

\section{Related Works}

% related works should be related to the aforementioned challenges and motivate the proposed method.

\section{Methodology}
\subsection{Architecture}
% Describe the architecture of the proposed method/framework. Need to clearly describe every part of the architecture (with a figure). (I usually describe the architecture according to the data flow)

\subsection{Problem Formulation}
% For some popular or newly emerged problems, such as FDD/transfer learning/one-short learning/zero-shot learning/multi-view learning/multi-task learning/multi-modal learning-based problems, should formulate the problem according to the topic discussed in this paper, using mathematical descriptions and symbols.


\subsection{Methodology}
% Detailed mathematical functions and formulas

\section{System}
% For application papers, need to describe the system based on background knowledge of the application field.
% Other papers do not need.

\section{Experiment}

\subsection{Datasets}
% Describe the adopted datasets and use citations.
% For data mining papers, at least two datasets should be included according to your topic; for AI papers, datasets are usually publicly standardized datasets and should include more datasets from different fields to prove the generalization of the proposed algorithm.
% For data mining papers, it is important to summarize the details of the datasets via a table. List the numbers of samples, features, rates of anomalies, cases, etc.

\subsection{Experiment Set-up and Research Questions}
% Describe the arrangement of how to use the experiments to prove the effectiveness of the proposed method.
% Here is a good place to propose several research questions and design the experiment plan according to RQs.

\subsection{Data Pre-process and Model Parameters}
% Briefly introduce the pre-processing procedures if you have pre-processed the data, like feature selection, sampling, normalization, signal transformation, window size, sliding steps, etc.
% Lists some details about the model, like numbers of neural network layers and hidden units, learning rates, optimizers, number of iterations, etc.
% Lists the experimental hardware environment, like the CPU and GPU you used. Sometimes, you should also report the training and testing time.

\subsection{Baselines}
% List SOTA methods and use citations.
% Note that SOTA should include some popular classic and recent methods related to the proposed method. 

\subsection{Results}
% Use tables to summarise all the testing results and hight the best (and sometimes second best) performance.
% Use plots to show the results from another angle, with more details.
% Analyzing the results and explaining why the proposed method cannot beat SOTA in some cases is important.
% For data mining papers, presenting your insights based on numerical results is important. For example, for the multi-class classification results, presenting the confusion matrix is usually essential to see which classes are confused by the classifier and try to analyze possible reasons by recalling the practical relationships and correlations among the confused classes.
% For AI papers, connecting numerical results with your motivations for improving this algorithm/method is important. Point out the evidence supporting your proposed method as a better solution for the problem discussed in this paper.

\subsection{Ablation Experiments}
% Ablation experiments are important to prove some modules of the proposed method are essential.

\section{Conclusion}
% 1-3 sentences to summarize what you have done in this paper. Mention the name of the proposed method, which can reflect the general framework, and highlight the core contributions (the way you improve the model/algorithm or the special angle you proposed to rethink the discussed topic and your solution). For example, "In this paper, we proposed xxx method, which can be used to solve xxx problem. It is based on XXX model and XXX technique."
% 1-2 sentences to summarize the experimental results (similar to what has been done in the Abstract section).
% 1-2 sentences to mention possible future works, such as extensions to other fields for data mining work or considering more conditions in future works. Note that this is not the place to judge your work. Never describe your work negatively.

\end{document}